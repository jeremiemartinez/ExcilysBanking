\documentclass[11pt]{article}

\usepackage[utf8]{inputenc}
\usepackage[french]{babel}
\usepackage[T1]{fontenc}
\usepackage{multirow}
\usepackage[left=2cm,top=2cm,right=3cm,bottom=3cm]{geometry}

\title{Projet d'application\\Excilys Banking}
\date{12/06/2012}

\begin{document}

\maketitle

\begin{center}
\begin{tabular}{ | c | c | c | c | }
\hline \textbf{Version} & \textbf{Édité par} & \textbf{Date} & \textbf{Notes} \\
\hline 1.00 & Luc Ponnau \& Jérémie Martinez & 13/06/2012 & Création \\
\hline & & & \\
\hline & & & \\ \hline
\end{tabular}
\end{center}

\section{Introduction}

Ce document décrit le projet d'application \emph{Excilys Banking} mené au sein
de l'entreprise \emph{EBusiness Information}. Ce projet, développé uniquement
par des stagiaires, marque la fin de la période de formation et permet de
valider la maîtrise de technologies telles que \emph{Spring} ou \emph{Hibernate},
ainsi que des méthodes agiles. Le projet repose sur la création d'un site
d'e-banking à destination d'une banque fictive.

\section{Description du projet}

\subsection{But}

Le but du projet est de pratiquer la méthode de développement \emph{Scrum} sur
une période de six itérations au sein d'une équipe de développement
restreinte et sur un projet mettant en \oe{}uvre
des technologies phares du monde Java/JEE.\\

Il est attendu de délivrer les fonctionnalités choisies par le \emph{Product Owner}
à chaque fin d'itération et finalement l'ensemble des fonctionnalités définies
dans le \emph{Scrum backlog} en fin de projet.\\

L'équipe fournira également des rapports de tests et une documentation permettant la prise en main de l'environnement de développement.

\subsection{Participants}

Le projet est développé au sein d'\emph{EBusiness Information} par
Jeremie Martinez et Luc Ponnau, et encadré par Stéphane Landelle, directeur
technique de l'entreprise. 

\subsection{Spécifications}

Les spécifications de ce projet sont précisées dans le \emph{Scrum Backlog} qui est contenu dans le répertoire \emph{doc} du projet.

\section{Organisation du projet}

\subsection{Organisation temporelle}

L'équipe de développement comprend deux personnes. Les journées de travail
commencent à 9h30 et finissent à 18h, avec une pause d'une heure et demie
pour le déjeuner.\\

Le processus de développement repose sur la méthode agile \emph{Scrum} et suit les
principes du développement dirigé par les tests. Cependant, la durée des itérations ou sprints se feront suivant le modèle de l'eXtreme Programming (XP), c'est-à-dire d'une semaine chacune et plus précisément du mardi au mardi. Des réunions de planification auront lieu
chaque mardi matin à 10h, des \emph{Daily scrums} chaque matin à 10h et des
réunions de livraison et rétrospective chaque lundi à 16h. Les réunions
se déroulent selon les recommandations de la méthode \emph{Scrum}.\\

\subsection{Planning et livraisons}

Le projet débute le 12 juin 2012. Une livraison est attendue à chaque fin de
sprint et le projet est prévu pour s'étaler sur six sprints de developpement
plus un sprint d'initialisation. La livraison finale de l'ensemble des
fonctionnalités attendues et donc prévue pour le 31 juillet 2012.

\begin{center}
\begin{tabular}{ | c | c | }
\hline \textbf{Date} & \textbf{Livraison} \\
\hline 26/06/2012 & Iteration 1 \\
\hline 03/07/2012 & Iteration 2 \\
\hline 10/07/2012 & Iteration 3 \\
\hline 17/07/2012 & Iteration 4 \\
\hline 24/07/2012 & Iteration 5 \\
\hline 31/07/2012 & Livraison finale \\ \hline
\end{tabular}
\end{center}

\subsection{Définition des Rôles}

L'équipe de développement est consituée de Jérémie Martinez et Luc Ponnau.
Stéphane Landelle a le double rôle de \emph{Product Owner} et de référant 
technique.

\begin{center}
\begin{tabular}{ | c | c | }
\hline \textbf{Role} & \textbf{Nom} \\
\hline Product Owner & Stéphane Landelle \\
\hline Scrum Master & Luc Ponnau \\
\hline \multirow{2}{*}{Developpeur} & Jérémie Martinez \\ & Luc Ponnau \\
\hline Référant Technique & Stéphane Landelle \\ \hline
\end{tabular}
\end{center}

\subsection{Organisation Scrum}

Chaque développeur travaille sur le poste qui lui a été assigné lors de son
arrivée. Les réunions de plannification, de livraison et de rétrospective
ont lieu dans la salle de réunion de l'open space.\\

Les tâches définies pour chaque itération sont représentées par des papiers
\emph{post-it} sur lesquels sont inscrits le nom de la tâche ainsi que son
estimation. Une fois la tache affectée à un développeur, 
il doit ajouter ses initiales sur le post-it correspondant.\\

Afin de suivre l'avancement du développement, nous utiliserons un tableau appelé \emph{Spring Backlog} et qui possèdera respectivement les colonnes suivantes: Functionality, Todo, Dev, Done. Ce tableau sera dessiné sur un tableau blanc et mis à jour à chaque \emph{Daily Scrum}.

\subsection{Fini fini}

Une tâche est considérée finie finie lorsque les conditions suivantes sont
réalisées:

\begin{itemize}
\item le code implémentant la tâche est écrit,
\item la conception liée à la tâche est satisfaisante,
\item la tâche est commentée et documentée si nécessaire,
\item le code suit les règles de codage imposées,
\item le code compile,
\item le code passe les tests unitaires,
\item le code est déployé avec succès,
\item le code passe les tests d'intégration,
\item le code est ajouté au gestionnaire de version.
\end{itemize}

\section{Documentation}

La totalité des documents associés au projet, à l'exception des rapports de tests, seront placé dans le répertoire \emph{doc} du projet. Les outils privilégies seront \LaTeX{} et \emph{LibreOffice Calc}. Ils seront maintenu par l'équipe projet. Ce document devra être mis à disposition directement en téléchargement sur le site \emph{github} du projet\footnote{https://github.com/lponnau/ExcilysBanking/downloads}.

\subsection{Scrum backlog}

Le \emph{Scrum backlog} contient l'ensemble des histoires définies pour
l'application ainsi que leurs coûts et valeurs estimées. Il ne peut être modifié
que par le \emph{Product Owner}. L'équipe de développement intervient seulement
lors de l'estimation des coûts. Comme précisé précedemment, le \emph{Scrum Backlog} est contenu dans le répertoire \emph{doc} du projet.

\subsection{Historique de planning}

L'historique de planning retrace l'ensemble des sprints déjà réalisés. Il est une archive du \emph{Sprint Backlog} afin de pouvoir garder une trace de l'avancement du projet sprint après sprint. Il sera réalisé au format \emph{xls} et placé dans le répertoire \emph{doc} du projet.

\subsection{Rapports}

L'équipe de développement produit des rapports lors de l'exécution des phases
de tests unitaires et d'intégration. Ces rapports formattés automatiquement
par \emph{Maven} sous forme de pages HTML sont intégrés au site \emph{Maven} du
projet\footnote{url à définir.}. Le site Maven contiendra également la \emph{Javadoc} ainsi que les rapports produits par le plugin \emph{checklist}.


\section{Environnement de développement}
\label{env-dev}

\subsection{Développement Java}

\subsection{Base de données}

\subsection{Serveurs}

\subsection{Tests unitaires}

\subsection{Gestion du code source}

\subsection{Intégration}

\subsection{Déploiement}

\subsection{Automatisation avec Maven}

\section{Normes}

\subsection{Langage}

\subsection{Outils de normalisation}

\subsection{Règles}

formatter eclipse, saveaction, english writting

\section{...}

\end{document}
