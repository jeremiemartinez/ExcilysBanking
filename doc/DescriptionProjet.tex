\documentclass[11pt]{article}

\usepackage[utf8]{inputenc}
\usepackage[french]{babel}
\usepackage[T1]{fontenc}
\usepackage{multirow}

\title{Projet d'application\\Excilys Banking}
\date{12/06/2012}

\begin{document}

\maketitle

\begin{center}
\begin{tabular}{ | c | c | c | c | }
\hline \textbf{Version} & \textbf{Édité par} & \textbf{Date} & \textbf{Notes} \\
\hline 1.0 & Luc Ponnau & 12/06/2012 &  \\
\hline & & & \\
\hline & & & \\
\hline & & & \\
\hline & & & \\ \hline
\end{tabular}
\end{center}

\section{Introduction}

Ce document décrit le projet d'application \emph{Excilys Banking} mené au sein
de l'entreprise \emph{EBusiness Information}. Ce projet, développé uniquement
par des stagiaires, marque la fin de la période de formation et permet de
valider la maîtrise de technologies telles que \emph{Spring} ou \emph{Hibernate},
ainsi que des méthodes agiles. Le projet repose sur la création d'un site
d'e-banking à destination d'une banque fictive.

\section{Description du projet}

\subsection{But}

Le but du projet est de pratiquer la méthode de développement \emph{Scrum} sur
une période de six itérations au sein d'une équipe de développement
restreinte et sur un projet mettant en \oe{}uvre
des technologies phares du monde Java/JEE: \emph{Hibernate} et \emph{Spring}.\\

Il est attendu de délivrer les fonctionnalités choisies par le \emph{Product Owner}
à chaque fin d'itération et finalement l'ensemble des fonctionnalités définies
dans le \emph{Scrum backlog} en fin de projet.\\

L'équipe fournira également des rapports et une documentation complète du
processus de développement.

\subsection{Personnel}

Le projet est développé au sein d'\emph{EBusiness Information} par
Jeremie Martinez et Luc Ponnau, et encadré par Stéphane Landelle, directeur
technique de l'entreprise. 

\subsection{Spécification}

ebanking

\subsection{Méthodes et technologies employées}

Le processus développement repose sur la méthode agile \emph{Scrum} et suit les
principes du développement dirigé par les tests.\\

Les technologies suivantes sont mises en \oe{}uvre dans le projet:

\begin{center}
\begin{tabular}{ | c | c | }
\hline Java & Java6, Spring, Hibernate \\
\hline Test & JUnit \\
\hline Log & Slf4j, Logback \\
\hline Compilation & Maven \\
\hline Sources & Git \\
\hline Base de données & PostgreSQL \\
\hline Serveur & Tomcat \\
\hline Integration & Jenkins \\
\hline Documentation & \LaTeX \\ \hline
\end{tabular}
\end{center}

\subsection{Organisation du projet}

L'équipe de développement comprend deux personnes. Les journées de travail
commencent à 9h30 et finissent à 18h, avec une pause d'une heure et demie
pour le déjeuner.\\

Le processus de développement est décomposé en sprints \emph{Scrum} d'une semaine
chacun, allant du mardi au mardi. Des réunions de planification auront lieu
chaque mardi matin à 10h, des \emph{Daily scums} chaque matin à 10h et des
réunions de livraison et rétrospective chaque lundi à 16h. Les réunions
se déroulent selon les recommandations de la méthode \emph{Scrum}.\\

\subsection{Planning et livraisons}

Le projet débute le 12 juin 2012. Une livraison est attendue à chaque fin de
sprint et le projet est prévu pour s'étaler sur six sprints de developpement
plus un sprint d'initialisation. La livraison finale de l'ensemble des
fonctionnalités attendues et donc prévue pour le 31 juillet 2012.

\begin{center}
\begin{tabular}{ | c | c | }
\hline \textbf{Date} & \textbf{Livraison} \\
\hline 26/06/2012 & Iteration 1 \\
\hline 03/07/2012 & Iteration 2 \\
\hline 10/07/2012 & Iteration 3 \\
\hline 17/07/2012 & Iteration 4 \\
\hline 24/07/2012 & Iteration 5 \\
\hline 31/07/2012 & Livraison finale \\ \hline
\end{tabular}
\end{center}

\section{Documentation}

La totalité des documents associés au projets, à l'exception des rapports de
tests, sont rédigés sous forme de documents \LaTeX et maintenus par l'équipe de
développement. Les documents compilés sous forme PDF sont disponibles en
téléchargement sur le site \emph{github} du projet\footnote{https://github.com/lponnau/ExcilysBanking/downloads}.

\subsection{Scrum backlog}

Le \emph{Scrum backlog} contient l'ensemble des histoires définies pour
l'application ainsi que leurs coûts et valeurs estimées. Il ne peut être modifié
que par le \emph{Product Owner}. L'équipe de développement intervient seulement
lors de l'estimation des coûts.

\subsection{Historique de planning}

L'historique de planning retrace l'ensemble des sprints réalisés et en cours de
réalisation. Il contient la définition des tâches à réaliser pour implémenter
chaque histoire ainsi que leurs statut et le compte rendu des réunions
hebdomadaires.

\subsection{Rapports}

L'équipe de développement produit des rapports lors de l'exécution des phases
de tests unitaires et d'intégration. Ces rapports formattés automatiquement
par \emph{Maven} sous forme de pages HTML et intégrés au site \emph{Maven} du
projet\footnote{url à définir.}.

\section{Modèle Scrum}

\subsection{Définition des Rôles}

L'équipe de développement est consituée de Jérémie Martinez et Luc Ponnau.
Stéphane Landelle a le double rôle de \emph{Product Owner} et de référant 
technique.

\begin{center}
\begin{tabular}{ | c | c | }
\hline \textbf{Role} & \textbf{Nom} \\
\hline Product Owner & Stéphane Landelle \\
\hline Scrum Master & Luc Ponnau \\
\hline \multirow{2}{*}{Developpeur} & Jérémie Martinez \\ & Luc Ponnau \\
\hline Référant Technique & Stéphane Landelle \\ \hline
\end{tabular}
\end{center}

\subsection{Fonctionnement du projet}

\subsubsection{Organisation}

Chaque développeur travaille sur le poste qui lui a été assigné lors de son
arrivée. Les réunions de plannification, de livraison et de rétrospective
ont lieu dans la salle de réunion de l'open space. Dans le cas où la salle
n'est pas disponible, les réunion pouront être repoussées de quelques 

Les tâches définies pour chaque itération sont représentées par des papiers
\emph{post-it} sur lesquels sont inscrits le nom de la tâche ainsi que son
estimation en nombre d'heures. Les 

\subsubsection{Plannification}

\subsubsection{Déroulement des sprints}

\subsubsection{Fini fini}

\subsubsection{Planning}

\section{Méthode de développement}

\subsection{Tests unitaires}

\subsection{Environnement de développement}

\subsubsection{Développement Java}

\subsubsection{Base de données}

\subsubsection{Serveurs}

\subsection{Gestion du code source}

\subsection{Intégration}

\subsection{Déploiement}

\subsection{Automatisation avec Maven}

\section{Normes}

\section{Normalisation du code Java}

\section{...}

\end{document}
